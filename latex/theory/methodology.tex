\section{Методика проведения расчетов}
Расчет параметров работы каскада колонн в режимах без отбора и с отбором 
проводится в следующей последовательности.

1. Рассчитывается значение коэффициента разделения для заданной 
температуры по формуле (\ref{eq:koeff_separation}):
\begin{equation}\label{eq:koeff_separation}
\alpha = 1 + \dfrac{4755}{T^{2}} - \dfrac{0,803}{T},
\end{equation}
\noindent где $T$ - температура, К.

2. По формуле (\ref{eq:koeff_enrichment}) рассчитывается коэффициент обогащения:
\begin{equation}\label{eq:koeff_enrichment}
\varepsilon = \alpha - 1
\end{equation}

3. Определяется минимальное число теоретических тарелок в обогатительной 
и регенеративной частях:
\begin{equation}
n_{\text{обог}} = \dfrac{1}{\varepsilon}\ln\dfrac{c_{P}(1-c_{F})}{c_{F}(1-c_{P})},
\end{equation}
\begin{equation}
    n_{\text{рег}} = \dfrac{1}{\varepsilon}\ln\dfrac{c_{F}(1-c_{W})}{c_{W}(1-c_{F})},
    \end{equation}
\noindent где $c_{P}, c_{W}, c_{F}$ - концентрации отбора, отвала и питания.

4. По формуле (\ref{eq:number_theoretical_plates}) вычисляется число 
теоретических тарелок:
\begin{equation}\label{eq:number_theoretical_plates}
n = 2(n_{\text{обог}} + n_{\text{рег}})
\end{equation}

5. Определяется количество колонн:
\begin{equation}
n_{\text{кол}}^{all} = \dfrac{n}{N},
\end{equation}
\noindent где $N$ - количество теоретических тарелок в одной колонне.

6. Рассчитывается изменение концентрации целевого изотопа в безотбоном 
режиме ($P$ = 0) по колоннам каскада с помощью формулы (\ref{eq:concentration_change}):
\begin{equation}\label{eq:concentration_change}
c_{1}(n_{\text{кол}}) = \dfrac{\dfrac{c_{W}}{1-c_{W}}e^{\varepsilon Nn_{\text{кол}}}}{1 + \dfrac{c_{W}}{1-c_{W}}e^{\varepsilon Nn_{\text{кол}}}}
\end{equation}

7. Определяется величина начального потока при работе каскада с 
заданным отбором по формуле (\ref{eq:initial_flow}):
\begin{equation}\label{eq:initial_flow}
L_{\text{нач}} = kP\dfrac{c_{P} - c_{F}}{\varepsilon c_{F}(1 - c_{F})},
\end{equation}
\noindent где $P$ - поток отбора, моль/ч; $k$ - коэффициент для сшивки каскада по концентрации отвала. 
В первом приближении $k = 2$, далее в зависимости от полученной концентрации отвала вычисляется по формуле
$k = 2 \pm 0,0001i$ ($i$ - цикл итерации).

8. Рассчитывается средний поток для каждой колонны по формуле (\ref{eq:mean_flow}):
\begin{equation}\label{eq:mean_flow}
L(n_{\text{кол}}) = \dfrac{1}{2}L_{\text{нач}}(1 - r)^{Nn_{\text{кол}}}\cdot (1 + (1 - r)^{-N}),
\end{equation}
\noindent где $r$ - доля сокращения потока на одной теоретической тарелке.

9. Рассчитывается изменение концентрации целевого изотопа в режиме 
с отбором по колоннам каскада по формуле (\ref{eq:concentration_c2}):
\begin{equation}\label{eq:concentration_c2}
    c_{2}(n_{\text{кол}}) = \dfrac{x_{1} + \dfrac{x_{1} - c_{P}}{c_{P} - x_{2}} e^{Nn_{\text{кол}}\varepsilon (x_{1} - x_{2})}x_{2}}{1 + \dfrac{x_{1} - c_{P}}{c_{P} - x_{2}} e^{Nn_{\text{кол}}\varepsilon (x_{1} - x_{2})}}
\end{equation}

\noindent где $x_{1,2} = \dfrac{1}{2}(1 + \dfrac{P}{L\varepsilon})\pm \sqrt{\dfrac{1}{4}(1 + \dfrac{P}{L\varepsilon})^{2} - \dfrac{P}{L\varepsilon}c_{P}}.$

10. Определяются величины потоков питания $F$ и отвала $W$ в каскаде для 
режима с отбором из системы уравнений (\ref{eq:system_material}):
\begin{equation}\label{eq:system_material}
\begin{cases}
Fc_{F} = Pc_{P} + Wc_{W}\\
F = P + W
\end{cases}
\end{equation}

11. Строятся графики изменения концентрации целевого изотопа в режимах 
без отбора и с отбором по колоннам каскада.

\newpage